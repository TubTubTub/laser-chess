\documentclass{article}
\usepackage{amsmath}
\usepackage{graphicx}
\usepackage{subcaption}
\usepackage{setspace}
\usepackage[backend=bibtex,style=verbose-trad2]{biblatex}
\usepackage{siunitx}
\usepackage{multirow}
\usepackage{booktabs}
\usepackage{longtable}
\usepackage{rotating}
\usepackage{pgfplotstable}
\usepackage{listings}
\usepackage{color}
\usepackage{hyperref}
\usepackage{enumitem}
 
\definecolor{codegreen}{rgb}{0,0.6,0}
\definecolor{codegray}{rgb}{0.5,0.5,0.5}
\definecolor{codepurple}{rgb}{0.58,0,0.82}
\definecolor{backcolour}{rgb}{0.95,0.95,0.92}

\lstset{ % General setup for the package
  backgroundcolor=\color{backcolour},   
  commentstyle=\color{codegreen},
  keywordstyle=\color{magenta},
  numberstyle=\tiny\color{codegray},
  stringstyle=\color{codepurple},
  basicstyle=\footnotesize,
  breakatwhitespace=false,         
  breaklines=true,                 
  captionpos=b,                    
  keepspaces=true,                 
  numbers=left,                    
  numbersep=5pt,                  
  showspaces=false,                
  showstringspaces=false,
  showtabs=false,                  
  tabsize=2
}
\lstset{
  language=Python
}

\sisetup{
    round-mode=places,
    round-precision=2,
}

\bibliography{learn}

\setcounter{tocdepth}{3}

\title{Learning LaTeX - a perilous journey}
\date{9/2/2025}
\author{Toby Mok}

\begin{document}

% \addtocontents{toc}{\setcounter{tocdepth}{1}}
% \section{Another section}
% \subsection{Subsection}
% \subsubsection{Subsubsection}
% \addtocontents{toc}{\setcounter{tocdepth}{3}}

\maketitle
\pagenumbering{gobble}

\newpage
\doublespacing
\tableofcontents
\singlespacing

\newpage
\pagenumbering{arabic}

\title{The beginning}

\section{First document}
Hello World!

\section{Sections}
\subsection{Subsection}
Structuring a document is easy!

\subsubsection{Subsubsection}
More text.

\paragraph{Paragraph}
Some more text.

\subparagraph{Subparagraph}
blah blah blah

\section{Packages}
guh

\begin{equation*}
f(x) = x + 1 * x^2
\end{equation*}

\section{Math}
Inline maths - $f(x) = log(x)$, wow!

\begin{align*}
1 + 2 &= 3\\
1 &= 3 - 2
\end{align*}

$\left(\frac{1}{\sqrt{x}}\right)$

\begin{equation*}
\left[
\begin{matrix}
1 & 0\\
0 & 1
\end{matrix}
\right]
\end{equation*}

\section{Figures}

\begin{figure}[h!]
\centering

\begin{subfigure}[b]{0.3\linewidth}
\includegraphics[width=\linewidth]{assets/solar.png}
\caption{A solar.}
\end{subfigure}
\begin{subfigure}[b]{0.3\linewidth}
\includegraphics[width=\linewidth]{assets/solar.png}
\caption{Another solar.}
\end{subfigure}
\begin{subfigure}[b]{0.3\linewidth}
\includegraphics[width=\linewidth]{assets/solar.png}
\caption{Yet another solar.}
\end{subfigure}
\begin{subfigure}[b]{0.5\linewidth}
\includegraphics[width=\linewidth]{assets/solar.png}
\caption{Too much solar.}
\end{subfigure}

\caption{The same Solar. two times.}
\label{fig:solar1}
\end{figure}

Figure \ref{fig:solar1} shows a Solar.

\section{Table of contents}
\begin{table}
\caption{Dummy table}
\end{table}

% \begin{appendix}
% \listoffigures
% \listoftables
% \end{appendix}

\section{BibTeX}
Citation embedded in text: \autocite[1]{DUMMY:1}
Second citation embedded in text: \autocite[1]{WEBSITE:1}
\printbibliography

\section{Footnotes}
This is an example test referring to the footnote\footnote{\label{feet}Hello feet}.
I'm referring to these dogs \ref{feet}.

\section{Tables}
\begin{longtable}[c]{l|S|r}
    \toprule
    \textbf{Value 1} & \textbf{Value2} & \textbf{Value 3}\\
    $\alpha$ & $\beta$ & $\gamma$\\
    \midrule
    \endfirsthead

    \textbf{Value 1} & \textbf{Value2} & \textbf{Value 3}\\
    $\alpha$ & $\beta$ & $\gamma$\\
    \midrule
    \endhead

    \multirow{2}{*}{1} & 2.35 & 3\\
    & 5.0 & 6\\
    3 & \multicolumn{2}{c|}{3.33}\\
    \multicolumn{3}{c|}{\multirow{2}{*}{1234}}\\
    \bottomrule
    
\caption{My first (second?) table.}
\label{tab:table1}
\end{longtable}

\begin{sidewaystable}
    \begin{center}
    \caption{Landscape table.}
    \label{tab:table2}
    \begin{tabular}{l|S|r}
        \toprule
        \textbf{Value 1} & \textbf{Value 2} & \textbf{Value 3}\\
        $\alpha$ & $\beta$ & $\gamma$ \\
      \midrule
      1 & 1110.1 & a\\
      2 & 10.1 & b\\
      3 & 23.113231 & c\\
      \bottomrule
    \end{tabular}
    \end{center}
\end{sidewaystable}

\section{Pgfplotstable}
\begin{table}[!ht]
\begin{center}
    
\pgfplotstabletypeset[
    multicolumn names, % allows to have multicolumn names
    col sep=comma, % the seperator in our .csv file
    display columns/0/.style={
      column name=$Value 1$, % name of first column
      column type={S},string type},  % use siunitx for formatting
    display columns/1/.style={
      column name=$Value 2$,
      column type={S},string type},  % use siunitx for formatting
    display columns/2/.style={
    column name=$Value 3$,
    column type={S},string type},
    every head row/.style={
      before row={\toprule}, % have a rule at top
      after row={
          \si{\ampere} & \si{\volt} & \si{\tesla}\\ % the units seperated by &
          \midrule} % rule under units
          },
      every last row/.style={after row=\bottomrule}, % rule at bottom
]{learn.csv}

\caption{Autogenerated table from .csv file.}
\label{table3}
\end{center}
\end{table}

\section{Code Listings}
\begin{lstlisting}
def hello_world():
  print("Hello world!")
\end{lstlisting}

\section{Hyperlinks}
This is my link: \href{http://www.latex-tutorial.com}{Latex tutorial}.
This is my URL: \url{http://www.latex-tutorial.com}
This is my email: \href{mailto:obama@yahoo.com}{obama@yahoo.com}

\section{Lists}
\begin{itemize}
\item One
\item Two
\item Three
\end{itemize}

\begin{enumerate}
\item One
\item Two
\item Three
\end{enumerate}

\begin{enumerate}
\item One
\begin{enumerate}
\item[--] Two
\item[$-$] Three
\item Four
\end{enumerate}
\item[$\ast$] Five
\item[a] Six
\end{enumerate}

\begin{enumerate}[label=(\roman*)]
\item One
\item Two
\item Three
\end{enumerate}

\end{document}